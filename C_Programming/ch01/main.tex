%%%%%%%%%%%%%%%%%%%%%%%%%%%%%%%%%%%%%%%%%
% Wenneker Article
% LaTeX Template
% Version 2.0 (28/2/17)
%
% This template was downloaded from:
% http://www.LaTeXTemplates.com
%
% Authors:
% Vel (vel@LaTeXTemplates.com)
% Frits Wenneker
%
% License:
% CC BY-NC-SA 3.0 (http://creativecommons.org/licenses/by-nc-sa/3.0/)
%
%%%%%%%%%%%%%%%%%%%%%%%%%%%%%%%%%%%%%%%%%

%----------------------------------------------------------------------------------------
%	PACKAGES AND OTHER DOCUMENT CONFIGURATIONS
%----------------------------------------------------------------------------------------

\documentclass[10pt, a4paper, twocolumn]{article} % 10pt font size (11 and 12 also possible), A4 paper (letterpaper for US letter) and two column layout (remove for one column)

\input{structure.tex} % Specifies the document structure and loads requires packages

%----------------------------------------------------------------------------------------
%	ARTICLE INFORMATION
%----------------------------------------------------------------------------------------

\title{C Programming} % The article title

\author{
	\authorstyle{Boitumelo Phetla} % Authors
}

% Example of a one line author/institution relationship
%\author{\newauthor{John Marston} \newinstitution{Universidad Nacional Autónoma de México, Mexico City, Mexico}}

\date{\today} % Add a date here if you would like one to appear underneath the title block, use \today for the current date, leave empty for no date

%----------------------------------------------------------------------------------------

\begin{document}

\maketitle % Print the title

\thispagestyle{firstpage} % Apply the page style for the first page (no headers and footers)

%----------------------------------------------------------------------------------------
%	ABSTRACT
%----------------------------------------------------------------------------------------

\lettrineabstract{C is a general-purpose, imperative computer programming language, supporting structured programming, lexical variable scope and recursion, while a static type system prevents many unintended operations.}

%----------------------------------------------------------------------------------------
%	ARTICLE CONTENTS
%----------------------------------------------------------------------------------------

\section{Code Snippets}

\subsection{Hello World}

\begin{lstlisting}
/*
* Author: Boitumelo Phetla
* How to compile on Terminal
* gcc -Wall -o hello_world hello_world.c
* ./hello_world
*/
#include <stdio.h>

int main(void){
	puts("Hello world!!");	//prints out to console screen
	return(100);	//returns anything
	}

\end{lstlisting}

\subsection{printf statement}

\begin{lstlisting}
/*
* Author: Boitumelo Phetla
* How to compile on Terminal
* gcc -Wall -o program program.c
* ./program
*/
#include <stdio.h>

int main(void){
	printf("Hello world!!\n");	//prints out to console screen
	return(0);
	}
\end{lstlisting}

\subsection{scanf statement}

\begin{lstlisting}
/*
* Author: Boitumelo Phetla
* How to compile on Terminal
* gcc -Wall -o program program.c
* ./program
*/
#include <stdio.h>

int main(void){
	int num = 0;	//initialization

	printf("Enter a number: ");
	scanf("%d", &num);
	printf("Number: %d\n", num);
	return(0);
	}
\end{lstlisting}

\subsection{Simple arithmentic Algorithm}

Calculate temperature in Celsius from Farenheit inputs.
$C = \frac{5}{9}(F - 32)$
\begin{lstlisting}
/*
temp/
		|__________celsius.h
		|__________temp.c
*/

/*celsius.h*/

#ifndef celsius_h
#define celsius_h

//C = 5/9 * (F - 32)
float cTemp(float k){
  return (9/5 * (k - 32));
}

#endif


/*temp.c*/

#include "celsius.h"
#include <stdio.h>

int main(void){

  float k[] = {100.1, 99.9, 88.8, 77.7, 66.6, 55.5, 44.4, 33.3, 22.2, 11.1, 5.55};

  for(int i = 0; i < sizeof(k)/sizeof(int); i++){
    	printf("%.2f k = %.2f C\n", k[i], cTemp(k[i]));
  }

  return 0;

}



/*Output*/

100.10 k = 68.10 C
99.90 k  = 67.90 C
88.80 k  = 56.80 C
77.70 k  = 45.70 C
66.60 k  = 34.60 C
55.50 k  = 23.50 C
44.40 k  = 12.40 C
33.30 k  = 1.30 C
22.20 k  = -9.80 C
11.10 k  = -20.90 C
5.55 k   = -26.45 C
\end{lstlisting}


\subsection{Preprocessor}

\begin{lstlisting}
#include <stdio.h>
#include <math.h> //library header file
#define PI 3.1415

//A = PI^2 * r
double area(double radius);

int main(void){
  double r = 1.00000388488484884453434343;
  printf("Area(%f) = %.2f\n", r, area(r));
  return(0);
}


double area(double radius){
  return pow(PI, 2) * radius; //math
}

/*Output*/
Area(1.000004) = 9.87
\end{lstlisting}

\newpage
\subsection{Do-While Statement}

\begin{lstlisting}
#include <stdio.h>
#include <math.h> //library header file
#define PI 3.1415

//A = PI^2 * r
double area(double radius);

int main(void){

  double r = 1.00000388488484884453434343;
  do{
      printf("Area(%f) = %.2f\n", r, area(r)); //executes nonetheless
  }while(r < 1); //terminates here condition not met

  return(0);
}

double area(double radius){
  return pow(PI, 2) * radius; //math
}

/*Output*/
Area(1.000004) = 9.87

\end{lstlisting}

\subsection{While statement}

\begin{lstlisting}
#include <stdio.h>

int main(void){

  int start = 0, stop = 100, stride = 10;
  int count = 0;
  while(start <= stop){
    printf("%d\t:\t%d\n", count, start);
    count+=1;
    start+=stride;
  }
  return 0;
}

/*Output*/

0       :       0
1       :       10
2       :       20
3       :       30
4       :       40
5       :       50
6       :       60
7       :       70
8       :       80
9       :       90
10      :       100
\end{lstlisting}

\newpage
\subsection{Constant, While, If Statement}

\begin{lstlisting}
include <stdio.h>
#include <math.h>

#define C 299792458       //speed of light (m/s)


float e(float m);         //e = mc^2

int main(void){

  //define sentinel as m= -1
  printf("To terminate the programe enter [-1]\n");

  float m = 0.0;

  printf("Enter mass [kg]: ");
  scanf("%f", &m);

  while(m > 0){
      printf("m = %.2f kg, e = %.10e m/s\n", m, e(m));
      printf("To terminate the programe enter [-1]\n");
      printf("Enter mass [kg]: ");
      scanf("%f", &m);
      if(m < 0){
        printf("Program terminated\n");
      }
  }
  return 0;
}

float e(float m){
  return (m*pow(C,2));
}


/*Output*/
To terminate the programe enter [-1]
Enter mass [kg]: 20
m = 20.00 kg, e = 1.7975103338e+18 m/s
To terminate the programe enter [-1]
Enter mass [kg]: -1
Program terminated
\end{lstlisting}

\subsection{Simple getchar putchar statements}

\begin{lstlisting}
#include <stdio.h>
int main(){
  char c = getchar(); //input
  putchar(c);         //display
  puts("");
  return 0;
}

/*Output*/
A
A
\end{lstlisting}

\subsection{Array of chars}

\begin{lstlisting}
#include <stdio.h>

int main(){
  int c;
  c = getchar();
  while(c != EOF){  //ctrl + D or Z
    putchar(c);
    c = getchar();
  }

  return 0;
}
\end{lstlisting}

\subsection{Main function without a type}

\begin{lstlisting}
#include <stdio.h>

main(){
  printf("Testing\n");
  return 0;
}

/*Output*/
main.c:3:1: warning: type specifier missing, defaults to 'int' [-Wimplicit-int]
main(){
^
1 warning generated.
Testing
\end{lstlisting}

\subsection{Static variables}

\begin{lstlisting}
#include <stdio.h>

/*
 * Static variables have a property of preserving
 * their value even after they are out of their scope.
 * static data_type variable_name = variable_value

	static variables

	static variables are allocated memory in data segment, not stack segment.

	1. data segment
	2. stack segment
	3. heap segment

	static variables are initialized as 0 in memory.
	static variables are used to eliminate scope of variables or functinos.

*/

int func();

int main(void){
	for(int i = 0; i < 5; i++){
			printf("Calling static method: %d\n", func());
	}
	return 0;
}

int func(){
	static int count = 0;
	count++;
	return count;
}

/*Output*/
Calling static method: 1
Calling static method: 2
Calling static method: 3
Calling static method: 4
Calling static method: 5
\end{lstlisting}

%----------------------------------------------------------------------------------------
%	BIBLIOGRAPHY
%----------------------------------------------------------------------------------------

\printbibliography[title={Bibliography}] % Print the bibliography, section title in curly brackets

%----------------------------------------------------------------------------------------

\end{document}
